%New colors defined below
\definecolor{codegreen}{rgb}{0,0.6,0}
\definecolor{codegray}{rgb}{0.5,0.5,0.5}
\definecolor{codepurple}{rgb}{0.58,0,0.82}
\definecolor{backcolour}{rgb}{0.95,0.95,0.92}

%Code listing style named "mystyle"
\lstdefinestyle{mystyle}{
  backgroundcolor=\color{backcolour}, commentstyle=\color{codegreen},
  keywordstyle=\color{magenta},
  numberstyle=\tiny\color{codegray},
  stringstyle=\color{codepurple},
  basicstyle=\ttfamily\footnotesize,
  breakatwhitespace=false,         
  breaklines=true,                 
  captionpos=b,                    
  keepspaces=true,                 
  numbers=left,                    
  numbersep=5pt,                  
  showspaces=false,                
  showstringspaces=false,
  showtabs=false,                  
  tabsize=2
}



\section{Trieda CreateFile(MRHS)}
Táto trieda slúži na vypísanie MRHS v štandardnom formáte. Pre vytvorenie objektu tohto typu je potrebné na vstup dať už vytvorený objekt typu MRHS. Trieda CreateFile obsahuje jednu funkciu s názvom create\_file. 

\subsection{Funkcia create\_file(file\_name)}
Vstup tejto funkcie, file\_name, je reťazec, ktorý predstavuje názov textového súboru (aj s príponou .txt). Táto funkcia po zavolaní vytvorí textový súbor s názvom file\_name a vypíše do neho MRHS sústavu z daného objektu typu CreateFile. \newline
%"mystyle" code listing set
\lstset{style=mystyle}
\begin{lstlisting}[language=Python, caption=Príklad volania funkcie create\_file(file\_name)]
from MRHS_Solver import CreateFile as cf

file = cf.CreateFile(mrhs)
file.create_file('output.txt')
\end{lstlisting}

\begin{lstlisting}[caption=Príklad výstupného textového súboru funkcie create\_file(file\_name)]
5
4
3 2
1 1
1 1
3 2
[1 0 0  0  1  1 0 0]
[0 1 0  0  1  0 1 0]
[0 0 0  1  1  0 0 1]
[0 0 0  0  0  1 0 0]
[0 0 1  0  1  1 1 1]
[1 1 1]
[1 0 1]

[0]

[1]

[1 0 1]
[0 1 0]
\end{lstlisting}